\documentclass{article}

\begin{document}
\title{Comparison of Epitopes and paratopes}

\author{Miruna Serian}
\date{2019}
\maketitle
\newpage 

\section*{Abstract}
\markright{}
This is the abstract
\section{Introduction}
\markright{}
\newpage 
\section{Methods}
\subsection{Identifying Ig/Ag complexes}
 72 Ig/Ag complexes associated with lysozyme where identified using IMGT's database. The antigens from the obtained complexes come from different species: Camelus dromedarius(14 complexes), shark - Ginglymostoma cirratum(4 complexes), mouse - Mus musculus (53 complexes). Three out of the complexes contain humanized antibodies.
 
\subsection{Retrieving contacts}
Thrugh data mining, annotation for the antibody and lysozyme chains for each pdb entry was obtained from the IMGT and was further used to collect information about the interactions between the antigen chain and both heavy and light chains from the antibody. The interacting residues' positions and their names were fetched from PDBSum and stored locally.

\subsection{Epitope identification}
The lysozyme chains from every complex were downloaded and their amino acid sequences were aligned using EBI' Clustal Omega MSA tool. The generated alignment was downloaded and a script was used to obtain the equivalent contact positions for every sequence which takes into account the insertions and gaps. The equivalent contact residues were then plotted onto a heatmap for first observations of the epitope. 


\section*{Discussion}

This is the section
\end{document}
